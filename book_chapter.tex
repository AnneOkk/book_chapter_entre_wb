\documentclass[a4paper,man,natbib]{apa6}

\usepackage[english]{babel}
\usepackage[utf8]{inputenc}
\usepackage{amsmath}
\usepackage{graphicx}
\usepackage[colorinlistoftodos]{todonotes}
\usepackage{natbib}

\title{Entrepreneurial Wellbeing}
\shorttitle{Entrepreneurial Wellbeing}
\author{Anne-Kathrin Kleine and Antje Schmitt}
\affiliation{University of Groningen}

\begin{document}

\abstract 

\maketitle

\section{Introduction}
\label{sec:examples}
A vibrant entrepreneurial scene adds to a nation’s economic development and welfare: Entrepreneurs create jobs and introduce innovative services and goods \citep{Acs.1988, Schumpeter.1934}. In addition to employment and welfare benefits, entrepreneurship may foster positive societal changes and the breakthrough of services and goods that fulfill social needs \cite[e.g.][]{Certo2008, Stephan2018, Wiklund2019}. \par 
For many people, one of the main reasons for choosing an entrepreneurial career is the high amount of control and autonomy associated with founding and leading one's own business: “Unlike most traditional occupations, entrepreneurs enjoy a level of freedom and control that can enable them to derive more meaning from their work, fulfill their innate talents and skills, and engage in purposeful activities through self-directed tasks” \cite[p.~580] {Wiklund2019}. However, these benefits do not come without costs: Entrepreneurs commonly have to cope with high amounts of uncertainty, responsibility, and risk and often perceive high work and time pressure \citep{Stephan2018}.  \par
Over the past decades, researchers have become more and more interested in disentangling the effects of environmental and person factors on entrepreneurial well-being \cite[e.g.][]{Stephan.2018, Shir.2019, Carter2011, Baron2008}. As psychological well-being is key to human functioning \citep{Aldwin1987}, fostering positive and limiting negative impacting factors of entrepreneurial well-being should be of high social interest. As has been highlighted by \cite{Wiklund2019}, people engage in entrepreneurial activities for various reasons. While performance-relevant aspects, such as financial benefits and impact on market development may be important for some people who decide for an entrepreneurial career, others may be driven by passion, the perspective of finding meaningfulness in their job and feeling good about themselves doing their work \citep[e.g.][]{Cardon2008, Aguinis2019}. In the end, being satisfied with one's job increases personal happiness and life satisfaction \cite[e.g.][]{WEAVER1978, Unanue2017} and people who feel satisfied and happy live longer \citep{Diener2011} and perform on higher levels \citep{VanDeVoorde2012}. \par
In their definition of well-being, researchers traditionally referred to the difference between hedonic and eudaimonic aspects of well-being \citep{Deci2008}. While the former refers to general feelings of happiness, the latter encompasses the feeling of “living well or actualizing one’s human potentials” \cite[p.~2]{Deci2008}. \cite{Wiklund2019} integrated both aspects in their definition of entrepreneurial well-being as “the experience of satisfaction, positive affect, infrequent negative affect, and psychological functioning in relation to developing, starting, growing, and running an entrepreneurial venture” (p. 582). That is, subjective entrepreneurial well-being may be understood in relation to the activities entrepreneurs engage in throughout their entrepreneurial career. One important point to consider when referring to entrepreneurial well-being is the distinction between the entrepreneur's subjective well-being \textit{while} this person engages in entrepreneurial activities and well-being that is \textit{derived from} the engagement in entrepreneurial activities \citep{Wiklund2019}. While considering context-specific measures of entrepreneurial well-being offers an interesting perspective on the rewards associated with an entrepreneurial career, broader measures of well-being may ''better reflect the individual experiences of well-being in this domain'' \cite[p.582] {Wiklund2019}. \par
Commonly, researchers have considered antecedents of entrepreneurial well-being in terms of personal factors, such as ..., environmental factors like ... and the effects of the engagement in different forms of behavior, like personal initiative \citep{Hahn2012}. \par 
In order to account for the interrelated influence of personality, environment, and behavior factors for both the emergence as well as the success entrepreneurial firm, researchers have adapted the organizational configuration approach to the entrepreneurial context (Covin & Slevin, 1991; Harms, Kraus, & Schwarz, 2009; Kraus, Kauranen, & Reschke, 2011; Wiklund & Shepherd, 2005). In essence, the configuration approach “suggest[s] the following structure of interrelated areas, which can be further separated into sets of aspects: characteristics of the (nascent)entrepreneurs, resources of the nascent entrepreneurs, environment, and organizing activities (management)” (Korunka, Frank, Lueger, & Mugler, 2003, p. 25). While a number of specific interrelated domains have been defined in the entrepreneurship literature, all of them may be broken down into the three broader categories personality, environment, and behavior (Dess, Lumpkin, & Covin, 1997; Harms et al., 2009; Korunka et al., 2003; Kraus et al., 2011). 


\section{Antecedents of Entrepreneurial Well-being}
\subsection{Personal Characteristics}
\subsection{Environmental Characteristics}
\section{A Process View on Entrepreneurship}
As we found out in the previous sections of this chapter, entrepreneurial well-being is affected by various factors. Personal characteristics, such as a.... may affect the entrepreneur's well-being through... . Environmental characteristics, such as ... play a crucial role for .... Finally, we learned that entrepreneurs typically engage in different forms of behaviors, which may be roughly summarized as exploratory and exploitative forms of behavior and that different behaviors may affect well-being in distinguishable ways. \par
In the following section of the book chapter, we want to discuss how these factors may be intertwined. For example, we will discover how the interaction between specific person characteristics and behaviors may affect entrepreneurial well-being or how environmental aspects may influence the impact of entrepreneurial behaviors on the entrepreneur's well-being. To better understand these effects on entrepreneurial well-being, we discuss the interactive effects of person characteristics, environmental characteristics, and behavior during the different phases of the entrepreneurial process. \par
Entrepreneurs pass through multiple distinguishable phases in the course of their entrepreneurial endeavor and each of these phases come along with distinct behavioral requirements \cite[e.g.][]{Baron.2002, vanGelderen.2005}. Researchers have attempted to classify and define the phases the entrepreneur (or the entrepreneurial firm) eventually passes. For example, \cite{Bosma.2019} suggest a division of the entrepreneurial career into (1), potential entrepreneurs who are concerned with the recognition of opportunities and the acquisition of knowledge and skills, (2) nascent entrepreneurs who are involved in setting up their business, (3) owner-manager of a new business (up to 3.5 years old), and (4), owner-manager of an established business (more than 3.5 years old). \cite{Rotefoss.2005} describe the activities entrepreneurs perform in different stages and propose a division corresponding to important milestones in the start-up process: First, the potential entrepreneur develops the intention to pursue an entrepreneurial career; second, they attempt to establish a business; and third, they establish their business. \cite{Rotefoss.2005} define three activities that are suitable for differentiating nascent entrepreneurs from early entrepreneurs, namely the investment of money, the receipt of the first payment, and the registration of the business as a legal entity (see also \cite{Reynolds.1992}. In other word, what distinguishes an entrepreneur from someone who might have developed entrepreneurial intentions, but who may not be defined as an entrepreneur yet, is the behavior this person engages in. In the following, we will zoom in on the interactive effects of person characteristics, environmental characteristics, and entrepreneurial behavior the entrepreneur's well-being during the entrepreneurial stages mentioned above. 
\subsubsection{Interactive Effects Throughout the Nascent Phase}
Entrepreneurial opportunities may be defined as situations in which new goods and services may be discovered, created, and exploited \citep{Venkataraman.2019}. Opportunity recognition is often defined as encompassing both cognitive as well as behavioral aspects. For example, according to \cite{Baron2006}, to recognize opportunities, entrepreneurs have to actively search for opportunities, stay alert to opportunities, and possess prior knowledge of the market to finally recognize opportunities when they come up. 



In the end, founding a business comes along with the necessity to 

When looking at the literature on entrepreneurial forms of behavior, it becomes clear that research on proactive forms of behavior dominates the field. When people think of entrepreneurs, they commonly imagine proactive individuals who engage in behaviors that include demonstrating initiative and creative thinking

 When referring to entrepreneurial forms of behavior, researchers commonly When focusing the behavioral aspect of entrepreneurship, researchers commonly refer to proactive forms of behavior as laying at the heart of entrepreneurship. For example, \cite{Hisrich1990} state that entrepreneurial behaviors ''include demonstrating initiative and creative thinking'' (p. 209). 

according to Proactivity is often considered as a core component of entrepreneurship in itself. Common definitions of entrepreneurship refer to 
While a proactive personality is commonly considered as one of the main antecedents entrepreneurial intentions \cite[e.g.][]{Crant1996} and the effects of proactive personality on entrepreneurial well-being have been explored extensively (ADD SOURCES), researchers have largely omitted to examine the effects of proactive behavior on entrepreneurial well-being (see \cite{Hahn2012} for an exception). 

entrepreneurs are commEntrepreneurs undoubtedly have to engage in exploratory forms of behaviors in order to exploit opportunities and realize innovative ideas \cite[e.g.,][]{Dess1999}. 
\cite{Hahn2012} assessed the impact of personal initiative, defined as self-starting, future-oriented and goal-directed action \citep{Bledow2009}, on 

\cite{Casciaro.2014}  Use section and subsection commands to organize your document. \LaTeX{} handles all the formatting and numbering automatically. Use ref and label commands for cross-references.

\subsection{Comments}

You can add inline TODO comments with the todonotes package, like this:
\todo[inline, color=green!40]{This is an inline comment.}

\subsection{References}

LaTeX automatically generates a bibliography in the APA style from your .bib file. The citep command generates a formatted citation in parentheses. The cite command generates one without parentheses. LaTeX was first discovered by.

\subsection{Tables and Figures}



% Commands to include a figure:
\begin{figure}
\centering
\caption{\label{fig:frog}This is a figure caption.}
\end{figure}



mik
You can make lists with automatic numbering \dots

\begin{enumerate}
\item Like this,
\item and like this.
\end{enumerate}
\dots or bullet points \dots
\begin{itemize}
\item Like this,
\item and like this.
\end{itemize}

We hope you find write\LaTeX\ useful, and please let us know if you have any feedback using the help menu above.

\bibliographystyle{apacite}
\bibliography{library}

\end{document}

%
% Please see the package documentation for more information
% on the APA6 document class:
%
% http://www.ctan.org/pkg/apa6
%