

\documentclass[a4paper,man,noextraspace,natbib]{apa6}

\usepackage[english]{babel}
\usepackage[utf8x]{inputenc}
\usepackage{graphicx}


\title{Entrepreneurial Wellbeing}
\shorttitle{Entrepreneurial Wellbeing}
\author{Anne-Kathrin Kleine and Antje Schmitt}
\affiliation{University of Groningen}

\abstract 

\begin{document}
\maketitle

\section{Introduction}
A vibrant entrepreneurial scene adds to a nation’s economic development and welfare: 
Entrepreneurs create jobs and introduce innovative services and goods \citep{Acs.1988, Schumpeter.1934}. 
In addition to employment and welfare benefits, entrepreneurship may foster positive societal changes and the breakthrough of services and goods that fulfill social needs \cite[e.g.][]{Certo2008, Stephan2018, Wiklund2019}. \par 

For many people, one of the main reasons for choosing an entrepreneurial career is the high amount of control and autonomy associated with founding and leading one's own business: 
“Unlike most traditional occupations, entrepreneurs enjoy a level of freedom and control that can enable them to derive more meaning from their work, fulfill their innate talents and skills, and engage in purposeful activities through self-directed tasks” \cite[p.~580] {Wiklund2019}. 
However, these benefits do not come without costs: 
Entrepreneurs commonly have to cope with high amounts of uncertainty, responsibility, and risk and often perceive high work and time pressure \citep{Stephan2018}. \par

Over the past decades, researchers have become more and more interested in disentangling the effects of environmental and person factors on entrepreneurial well-being \cite[e.g.][]{Stephan.2018, Shir.2019, Carter2011, Baron2008}. 
As psychological well-being is key to human functioning \citep{Aldwin1987}, fostering positive and limiting negative impacting factors of entrepreneurial well-being should be of high social interest. 
As has been highlighted by \cite{Wiklund2019}, people engage in entrepreneurial activities for various reasons. 
While performance-relevant aspects, such as financial benefits and impact on market development may be important for some people who decide for an entrepreneurial career, others may be driven by passion, the perspective of finding meaningfulness in their job and feeling good about themselves doing their work \citep[e.g.][]{Cardon2008, Aguinis2019}. 
In the end, being satisfied with one's job increases personal happiness and life satisfaction \cite[e.g.][]{WEAVER1978, Unanue2017} and people who feel satisfied and happy live longer \citep{Diener2011} and perform on higher levels \citep{VanDeVoorde2012}. \par

In their definition of well-being, researchers traditionally referred to the difference between hedonic and eudaimonic aspects of well-being \citep{Deci2008}. 
While the former refers to general feelings of happiness, the latter encompasses the feeling of “living well or actualizing one’s human potentials” \cite[p.~2]{Deci2008}. 
\cite{Wiklund2019} integrated both aspects in their definition of entrepreneurial well-being as “the experience of satisfaction, positive affect, infrequent negative affect, and psychological functioning in relation to developing, starting, growing, and running an entrepreneurial venture” (p. 582). 
That is, subjective entrepreneurial well-being may be understood in relation to the activities entrepreneurs engage in throughout their entrepreneurial career. 
One important point to consider when referring to entrepreneurial well-being is the distinction between the entrepreneur's subjective well-being \textit{while} this person engages in entrepreneurial activities and subjective well-being that is \textit{derived from} the engagement in entrepreneurial activities \citep{Wiklund2019}. 
While considering context-specific measures of entrepreneurial well-being offers insights into the rewards associated with an entrepreneurial career, broader measures of well-being may add to our understanding of the experience of well-being among entrepreneurs \cite[p.582] {Wiklund2019}. \par

Commonly, researchers have considered antecedents of entrepreneurial well-being in terms of personal factors, such as ..., environmental factors like ... and the effects of the engagement in different forms of behavior, like personal initiative \citep{Hahn2012}. \par 

Naturally, personal, environment, and behavioral variables do not influence the entrepreneur's well-being in isolation. 
The importance of understanding the interplay of these attributes is strongly advocated in entrepreneurial process models, which explain the venture creation process as a sequence of events that emerge from the interaction of specific environmental and individual attributes and the behavior the entrepreneur engages in \cite{Moroz2012}. For example (...) \par

In the following sections we first outline what entrepreneurship and entrepreneurial well-being encompass. 
Next, we explore the factors that influence entrepreneurial well-being and develop an integrative model of entrepreneurial well-being that is based on the understanding of entrepreneurship as a process. 
Finally, we will offer some implications for future research and conclude by summarizing implications for practice.

\subsection{Entrepreneurship and Entrepreneurial Well-being}

\subsubsection{Defining Entrepreneurship}

Defining entrepreneurship is a difficult endeavor. In a very broad sense, entrepreneurship refers to an occupational category. 
In this sense, entrepreneurs are all those who are either self-employed or owner-managers of a business as opposed to those who are employed by others or unemployed \citep{Gorgievski2016a, VanPraag2007}. 
Entrepreneurs are commonly differentiated from managers in terms of their emphasis on innovation \cite[e.g.][]{Drucker2014}. 
That is, in contrast to the latter, entrepreneurs create a novel business or service that did not exist before. Following this notion, entrepreneurs are those who engage in new venture creation \citep{Shane2000}. \par

\paragraph{The role of person, environment, and behavior factors}
If we want to gain a deeper understanding of the matter at hand, we may refer to different sub-fields within the entrepreneurship literature. 
Typically, the understanding of the term entrepreneurship depends heavily on the approach researchers take when investigating entrepreneurial businesses or individuals: 
First, entrepreneurship may be understood in terms of the individual characteristics of the entrepreneur \citep{Bird1988, Rauch2007, Obschonka2017}. 
This conceptualization stems from the notion that the entrepreneur's individual characteristics are what drives their behavior \citep{Cunningham1991}. 
What \cite{Cunningham1991} describe as "Psychological Characteristics School of Entrepreneurship" may be summarized as the idea that "entrepreneurs have unique values and attitudes toward work and life" (p. 48). 
- How a person becomes an entrepreneur and how this person acts within the role as an entrepreneur is greatly influenced by those needs, beliefs and values \cite[p.~48]{Cunningham1991}. 
Over the past decades, researchers have become increasingly interested in investigating the personality attributes of entrepreneurs. 
Oftentimes entrepreneurs have. For example, in their meta-analytic review, \cite{Zhao2006} compared managers and entrepreneurs on the Big Five personality dimensions and found that entrepreneurs scored higher on Conscientiousness and Openness to Experience and lower on Neuroticism and Agreeableness. 
\cite{Brandstatter2011} summarized findings of five meta-analyses investigating the personality aspects of entrepreneurs and summarized that commonly entrepreneurs to have a higher need for achievement, and are more risk prone and more extraverted than managers. 
In an attempt to integrate models of the entrepreneurial personality and to represent the dynamics involved, \cite{Obschonka2017} developed the entrepreneurial personality system. 
They differentiate between basic personality tendencies (such as the Big Five personality traits) that are consistent across situations and contexts and characteristic adaptations encompassing traits that are more context-dependent, that is, influenced by basic personality tendencies, the context, and the learning environment (such as risk-taking, internal locus of control, and entrepreneurial self-efficacy). 
Obschonka's \citeyearpar{Obschonka2017} model connects the most prevalent approaches of research on the entrepreneurial personality. 
Their model is an attempt to integrate relatively isolated views on the effects of specific personal characteristics on relevant outcomes. 
They point out that in order to fully capture how the entrepreneur's personality influences the entrepreneur's attitudes and behaviors \textit{both} general individual tendencies in terms of generic, stable traits as well as context-specific, malleable personal characteristics have to be taken into account. \par

\cite{Obschonka2017} emphasize that personality does not affect outcomes (like intentions or behavior) in a deterministic way. 
Rather, they assume a dynamic and reciprocal relationship between personality and outcomes, in which, for example, the achievement of a central outcome may affect the more malleable personal characteristics of the entrepreneur. 
A part of this notion encompasses the image of the entrepreneur as someone whose actions and reactions are embedded within a specific (entrepreneurial) context. 
While \cite{Obschonka2017} focus on the aspects of the entrepreneurial personality, there exists a long tradition of research on the environmental (i.e., context) aspects of entrepreneurship. \par

Researchers representing the ``context-view'' claim that entrepreneurship may be understood in terms of firm characteristics, such as firm size (e.g., small to medium), firm age (e.g., novel business), or ownership (e.g., individually owned) \citep{Audretsch2012}. 
\cite{Audretsch2012} point out that the set of organizational characteristics that define an entrepreneurial firm is not clear-cut. 
Drawing from the understanding of entrepreneurship as a contextual phenomenon, researchers have examined the impact of factors outside the entrepreneurial firm that may influence organization establishment and success, such as the social context in which the entrepreneur operates.  
For example, \cite{Acs2006} argue that all phases of the entrepreneurial process are embedded within a system of social interactions. 
In order to establish a business, entrepreneurs necessarily have to talk to stakeholders, customers, collaborators, or other supporters. 
The entrepreneur's social network, in turn, is embedded in an environmental system. 
\cite{Acs2014} emphasize the important role of the entrepreneurial context for both firm emergence and growth by stating that "the 'heroic' entrepreneur is not the sole determinant of entrepreneurial success, and [...] the environment - or the 'ecosystem' - can play an important role in nurturing new venture seeds into fully-fledged, value-adding growth ventures" (p. 490). 
They criticize that researchers have largely omitted the role of country-level aspects when studying entrepreneurial processes. 
In their 'National Systems of Innovation' approach, the authors provide a pathway towards understanding entrepreneurial outcomes as embedded within national-level contexts. 
This tenet fits findings of empirical studies showing that a country's monetary policy, social climate, or unemployment rate may influence innovation and entrepreneurship activities \cite[e.g.][]{Galindo2014, Marcotte2013, Spencer2004}. \par

Finally, entrepreneurship may be understood in terms of the actions the entrepreneur (or the entrepreneurial firm) performs \cite[e.g.][]{Gartner1988, Lumpkin1996, Covin1991}. 
The basic tenet of this stream of research is the notion that entrepreneurship as a whole may be understood as a form of behavior \citep{Gartner1988}. 
In opposition to the personality approach toward entrepreneurship, \cite{Gartner1988} note: 
"The personality characteristics of the entrepreneur are ancillary to the entrepreneur's behaviors. Research on the entrepreneur should focus on what the entrepreneur does and not who the entrepreneur is" \cite[p.~57]{Gartner1988}. 
They argue that without \textit{acting} upon an opportunity, the entrepreneurial firm would never have emerged \citep{Gartner1988}.
\cite{Gartner2005} explain the entrepreneurial behavior perspective on entrepreneurship: 
"[...] organizations are not created by their context [..] and it is through the actions of entrepreneurs that organizations come into existence" and individual characteristics" (p. 5). 
According to researchers approaching entrepreneurship from a behavioral perspective, individual characteristics matter in so far as they may be associated with differences in entrepreneurial behaviors \citep{Gartner2005}. 
Entrepreneurial behaviors may be studied in terms of the (sequential) set of actions that have to be performed when founding a business, such as purchasing raw material; hiring and training employees; and producing, distributing, and marketing products or services \citep{Gatewood1995}. 
Second, they may be examined in terms of the effects general behavioral tendencies, such as competitive and proactive, risk-taking, and innovative behaviors have on firm emergence and success \citep{Covin1991a}. 

\subsubsection{Defining Entrepreneurial Well-being}
In a general sense, subjective well-being is determined by people's cognitive and affective evaluation of their lives \citep{Diener2000}. 
According to \cite{Deci2008} subjective well-being may be operationalized as ``experiencing a high level of positive affect, a low level of negative affect, and a high degree ofsatisfaction with one's life'' (p. 1). 
\cite{Deci2008, Ryan2001} differentiate between two forms of subjective well-being: hedonia and eudaimonia. 
While the former is primarily concerned with feeling good and experiencing happiness, eudaimonia may be defined as realizing one's potential. 
Hedonic and eudaimonic well-being may be defined as two viewpoints on subjesctive well-being: 
Hedonic well-being may be equated with happiness, positive affect, the absence of negative affect, and greater life satisfaction. In contrast, the eudaimonic viewpoint focuses on psychological well-being, which is defined more broadly in terms of the fully functioning person. \par

According to \cite{Diener2002}, theories of the emergence of subjective well-being may be broken down to three general approaches. 
First, need and goal satisfaction theories, according to which the reduction of adversity and the satisfaction of needs determine the experience of subjective health and well-being. 
Second, process or activity theories, which presume that the engagement in a subjectively interesting and fulfilling acivity leads to personal fulfillment and well-being. And third, trait theories, accoording to which the experience of personal health and well-being may be best explained as a result of the impact of relatively stable personality dispositions and cognitions \citep{Diener2002}. \par

Following \cite{Keyes2002}, subjective well-being may be understood as a continuum \cite{Keyes2002}. 
At one end, we find subjective ill-being in terms of mental health problems like depressiveness and anxiety that impair a person's level of functioning. 
On the other end, we find the experience of happiness, satisfaction and joy, leading to high levels of functioning \citep{Keyes2002}. \par

In consideration of the general approaches towardunderstanding subjective well-being, we may derive three principles in relation to research on entrepreneurial well-being: 
First, experiences associated with hedonic and eudaimonic well-being may, but do not have to, emerge from the engagement in entrepreneurial activities \citep{Wiklund2019}.
Examples of general hedonic and eudaimonic well-being encompass positive affect, the absence of forms of negative affect, general life satisfaction, and personal growth, meaningfulness, and vitality respectively. 
Examples of context-specific measures of entrepreneurial well-being are job satisfaction, work-family balance, or indicators of occupational health \cite{Stephan2018}. 
According to \cite{Wiklund2019}, context-specific measures of entrepreneurial well-being offer insights into the rewards associated with an entrepreneurial career and considering both general and context-specific indicators of subjective well-being allows arriving at an accurate impression of the entrepreneur's well-being \citep{Wiklund2019}. \par 

Second, entrepreneurial well-being may be understood as reflecting a continuum ranging from ill-being, reflected by, for example, the experience of mental health issues, and the presence of negative affect to well-being in terms of the experience of, for example, happiness, personal growth, and functioning. \par

Finally, in accordance with theories on the emergence of entrepreneurial well-being, we may study antecedents of entrepreneurial well-being in terms of personal, environmental, and behavioral factors \citep{Diener2002}.

\subsection{Antecedents of Entrepreneurial Well-being}
In the following sections we will explore the factors that influence entrepreneurs' well-being. Specifically, we review findings of studies exploring personal characteristics, environmental factors, and forms of entrepreneurial behaviors as antecedents of entrepreneurial well-being. 

\subsubsection{Personal Characteristics}



\subsubsection{Environmental Characteristics}

\subsubsection{Behavioral Characteristics}

\subsection{A Process View on Entrepreneurship}
As we found out in the previous sections of this chapter, entrepreneurial well-being is affected by various factors. Personal characteristics, such as a.... may affect the entrepreneur's well-being through... . Environmental characteristics, such as ... play a crucial role for .... Finally, we learned that entrepreneurs typically engage in different forms of behaviors, which may be roughly summarized as exploratory and exploitative forms of behavior and that different behaviors may affect well-being in distinguishable ways. \par
In the following section of the book chapter, we want to discuss how these factors may be intertwined. For example, we will discover how the interaction between specific person characteristics and behaviors may affect entrepreneurial well-being or how environmental aspects may influence the impact of entrepreneurial behaviors on the entrepreneur's well-being. To better understand these effects on entrepreneurial well-being, we discuss the interactive effects of person characteristics, environmental characteristics, and behavior during the different phases of the entrepreneurial process. \par
Entrepreneurs pass through multiple distinguishable phases in the course of their entrepreneurial endeavor and each of these phases come along with distinct behavioral requirements \cite[e.g.][]{Baron.2002, vanGelderen.2005}. Researchers have attempted to classify and define the phases the entrepreneur (or the entrepreneurial firm) eventually passes. For example, \cite{Bosma.2019} suggest a division of the entrepreneurial career into (1), potential entrepreneurs who are concerned with the recognition of opportunities and the acquisition of knowledge and skills, (2) nascent entrepreneurs who are involved in setting up their business, (3) owner-manager of a new business (up to 3.5 years old), and (4), owner-manager of an established business (more than 3.5 years old). \cite{Rotefoss.2005} describe the activities entrepreneurs perform in different stages and propose a division corresponding to important milestones in the start-up process: First, the potential entrepreneur develops the intention to pursue an entrepreneurial career; second, they attempt to establish a business; and third, they establish their business. \cite{Rotefoss.2005} define three activities that are suitable for differentiating nascent entrepreneurs from early entrepreneurs, namely the investment of money, the receipt of the first payment, and the registration of the business as a legal entity (see also \cite{Reynolds.1992}. In other word, what distinguishes an entrepreneur from someone who might have developed entrepreneurial intentions, but who may not be defined as an entrepreneur yet, is the behavior this person engages in. In the following, we will zoom in on the interactive effects of person characteristics, environmental characteristics, and entrepreneurial behavior the entrepreneur's well-being during the entrepreneurial stages mentioned above. 

\subsection{Entrepreneurial Well-being: A Process View}

\section{Other ideas - please disregard!}
\subsubsection{Interactive Effects Throughout the Nascent Phase}
Entrepreneurial opportunities may be defined as situations in which new goods and services may be discovered, created, and exploited \citep{Venkataraman.2019}. Opportunity recognition is often defined as encompassing both cognitive as well as behavioral aspects. For example, according to \cite{Baron2006}, to recognize opportunities, entrepreneurs have to actively search for opportunities, stay alert to opportunities, and possess prior knowledge of the market to finally recognize opportunities when they come up. 

In the end, founding a business comes along with the necessity to 

When looking at the literature on entrepreneurial forms of behavior, it becomes clear that research on proactive forms of behavior dominates the field. When people think of entrepreneurs, they commonly imagine proactive individuals who engage in behaviors that include demonstrating initiative and creative thinking

 When referring to entrepreneurial forms of behavior, researchers commonly When focusing the behavioral aspect of entrepreneurship, researchers commonly refer to proactive forms of behavior as laying at the heart of entrepreneurship. For example, \cite{Hisrich1990} state that entrepreneurial behaviors ''include demonstrating initiative and creative thinking'' (p. 209). 

according to Proactivity is often considered as a core component of entrepreneurship in itself. Common definitions of entrepreneurship refer to 
While a proactive personality is commonly considered as one of the main antecedents entrepreneurial intentions \cite[e.g.][]{Crant1996} and the effects of proactive personality on entrepreneurial well-being have been explored extensively (ADD SOURCES), researchers have largely omitted to examine the effects of proactive behavior on entrepreneurial well-being (see \cite{Hahn2012} for an exception). 

entrepreneurs are commEntrepreneurs undoubtedly have to engage in exploratory forms of behaviors in order to exploit opportunities and realize innovative ideas \cite[e.g.,][]{Dess1999}. 

\cite{Hahn2012} assessed the impact of personal initiative, defined as self-starting, future-oriented and goal-directed action \citep{Bledow2009}, on 

\bibliography{library}

\end{document}

%
% Please see the package documentation for more information
% on the APA6 document class:
% http://www.ctan.org/pkg/apa6
%

